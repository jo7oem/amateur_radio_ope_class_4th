{\large 電気回路 - 受動素子 -}
\begin{multicols}{3}
    {\huge 抵抗 R }
    \begin{center}
    \begin{circuitikz}
        \draw (0,0) to[R] (2,0);
    \end{circuitikz}

    \begin{circuitikz}
        \draw (0,0) to[generic] (2,0);
    \end{circuitikz}
    \end{center}

    \begin{itemize}
        \item 炭素や長い導線、セメントなどで作る\newline 大きな抵抗値が必要な場合は磁器で作る => がいし
        \item 電流が流れるのを防ぐ \newline => 電気抵抗
        \item 電流を消費して熱にする \newline 1[W] = 1[J/s] \newline => 電力
    \end{itemize}
    \columnbreak

    {\huge インダクタンス L }
    \begin{center}
        \begin{circuitikz}
            \draw (0,0) to[L] (2,0);
        \end{circuitikz}
    \end{center}
    \vspace{0.3cm}

    \begin{itemize}
        \item コイルともいう
        \item 動線を巻いたコイル
        \item 巻く芯(磁芯)によって特性が変わる
        \item 磁力(磁束)の形でエネルギーを蓄える
        \item 原理はファラデーの電磁誘導の法則
        \item 直流と交流で異なる値
        \item 直流だと電線とみなせる
    \end{itemize}

    \columnbreak
    {\huge キャパシタンス L }
    \begin{center}
        \begin{circuitikz}
            \draw (0,0) to[C] (2,0);
        \end{circuitikz}
    \end{center}
    \vspace{0.1cm}

    \begin{itemize}
        \item コンデンサともいう
        \item 2枚の導体を電気的に絶縁して挟む
        \item 導体の間に挟まれるものが誘電体
        \item 電荷の形でエネルギーを蓄える
        \item 直流だと電流が流れない
        \item 交流だと電流が流れる
    \end{itemize}
\end{multicols}
\begin{table}[htbp]
    \caption{受動素子のまとめ}
    \label{tab:mem}
    \begin{center}
    \begin{tabular}{l|c|c|c} \hline
    名前&抵抗&インダクタンス&キャパシタンス\\ \hline
    [単位](読み方)&[Ω](オーム)&[H](ヘンリー)&[F](ファラド)\\ \hline
    記号&R&L&C\\ \hline
    記号の由来&抵抗率&リアクタンス&コンデンサ\\ \hline
    値を変える方法&長さ、直径、温度&巻き数、磁芯(コア)の材質&板の面積、板の間隔、誘電体の材料\\ \hline
    周波数を上げると&変わらない&大きくなる&小さくなる\\ \hline
    直流の場合&抵抗&0&無限大\\ \hline
    交流の場合&インピーダンス&リアクタンス&リアクタンス\\ \hline
    式& R &$X_L = 2\pi fL$ & $X_C = \frac{1}{2 \pi fc}$
    \end{tabular}
        \end{center}
    \here
\end{table}