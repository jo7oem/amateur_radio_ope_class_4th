{\large 電気回路 - オームの法則・回路の合成 -}


\begin{multicols}{3}
    \begin{table}[H]
        \caption{記号と単位}
        \label{tab:記号}
        \begin{center}
            \begin{tabular}{llcl} \hline
                記号 & 名称 & 単位&読み方 \\ \hline
                $V$ & 電圧 & [V] &ボルト\\
                $I$ & 電流 & [A] &アンペア\\
                $R$ & 抵抗 & [$\Omega$] &オーム\\
                $P$ & 電力 & [W]&ワット \\ \hline
            \end{tabular}
        \end{center}
    \end{table}

    \begin{itembox}[l]{オームの法則}
        \begin{center}
            \begin{circuitikz}
                \draw
                (0,0) to[short] (2,0)
                to[R,l=$R$,v=$V_R$] (2,2)
                (0,2)to[short,i=$I$] (2,2)
                (0,2)to[battery1,l_=$V$] (0,0)
            \end{circuitikz}
        \end{center}
        \begin{itemize}
            \item 電流$I$が流れるとき、抵抗$R$にかかる電圧$V_R$は、$V_R = IR$
            \item 抵抗で消費される電力$P$は\newline$P = IV_R = I^2R = V^2/R$
        \end{itemize}
    \end{itembox}

    \begin{itembox}[l]{直列回路}
        \begin{center}
            \begin{circuitikz}
                \draw
                (0,4)to[battery1,l_=$V$] (0,0)
                (0,4) to[short,i=$I$] (2,4)
                to[R,l_=$R_1$,v^<=$V_{R1}$] (2,2)
                to[R,l_=$R_2$,v^<=$V_{R2}$] (2,0)
                to[short] (0,0)
            \end{circuitikz}
        \end{center}
        \begin{itemize}
            \item 合成抵抗$R = R_1 + R_2$
            \item 合成電圧$V = V_{R1} + V_{R2}$ \newline(直列回路は素子ごとに電圧が違う)
            \item 合成電流$I = I_1 = I_2$\newline(直列回路は回路全体で電流が一定)
        \end{itemize}
    \end{itembox}

    \begin{itembox}[l]{並列回路}
        \begin{center}
            \begin{circuitikz}
                \draw
                (0,4) to[battery1,l_=$V$] (0,0)
                (0,4) to[short,i=$I$] (3,4)
                to[short,-*] (3,3)
                (4,3)to[short](2,3)
                (0,0) to[short] (3,0)
                (2,1) to[R,l=$R_1$,i<=$I_1$,v=$V_{R1}$] (2,3)
                (4,1) to[R,l=$R_2$,i<=$I_2$,v=$V_{R2}$] (4,3)
                (4,1)to[short](2,1)
                (3,0)to[short](3,1)
            \end{circuitikz}
        \end{center}
        \begin{itemize}
            \item 合成抵抗$1/R = 1/R_1 + 1/R_2$\newline
            $R = \dfrac{R_1R_2}{R_1 + R_2}$
            \item 合成電圧$V = V_1 = V_2$\newline(並列回路は回路全体で電圧が同じ)
            \item 合成電流$I = I_1 + I_2$\newline(並列回路は素子ごとに電流が違う)
        \end{itemize}
    \end{itembox}

\end{multicols}
\here
\begin{multicols}{2}
    \begin{table}[H]
        \caption{オームの法則のまとめ}
        \label{tab:オームの法則}
        \begin{center}
            \begin{tabular}{c|c|c} \hline
             $V$ & $I$ & $R$ \\ \hline
             && \\
             $V=IR$&$I=\dfrac{V}{R}$&$R=\dfrac{V}{I}$\\
            \end{tabular}
        \end{center}
    \end{table}
\begin{table}[H]
    \caption{素子の合成のまとめ}
    \label{tab:素子の合成}
    \begin{center}
        \begin{tabular}{lccc} \hline
         & R & L & C \\ \hline
        直列 & $R=R_1+R_2$ & $L=L_1+L_2$ & $C=\dfrac{C_1C_2}{C_1+C_2}$\\
        並列 & $R=\dfrac{R_1R_2}{R_1+R_2}$ & $L=\dfrac{L_1L_2}{L_1+L_2}$ & $C=C_1+C_2$\\
        \end{tabular}
    \end{center}
\end{table}
\end{multicols}
